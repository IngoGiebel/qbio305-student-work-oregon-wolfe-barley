\section{Results}

\subsection{Analysis of variance for the genotype effect on the plant height}

To investigate the association between the height of the plants and their genotype, we conducted a single-factor analysis of variance (ANOVA). The analysis revealed a highly significant impact of the genotype on the plant height (p-value < \num{2e-16} for both the raw and the cleansed data in which three outliers were removed). As the corresponding F-value for the cleansed data (\num{834.4}) is substantially greater than the corresponding F-value for the raw data (\num{198.1}), the actual p-value for the cleansed data is actually even much smaller than the p-value for the raw data. Figure \ref{fig:raw_genotype_height} shows density plots of the raw height data for the different genotypes.

\begin{figure}[htbp]
    \includesvg[width=\linewidth]{../results/raw_genotype_height}
    \caption{Density plots of the raw height data for the different genotypes}
    \label{fig:raw_genotype_height}
\end{figure}


Show the ANOVA table and an overview of the linkage and QTL plots you have generated in this course with the code you have used.

What are your major findings?

Are your traits in proximity to any of the following known genes and if so, can they explain your trait?

Locus Chromosome References:

VRS1 2H \url{https://doi.org/10.3390/ijms23042276}

ZEO1 2H \url{https://doi.org/10.1242/dev.170373}

Kap 4H \url{https://doi.org/10.1038/374727a0}

LKS2 7H \url{https://doi.org/10.1093/jxb/ers182}

Gpa1 2H \url{https://doi.org/10.1186/s12870-021-02915-9}

Nud 7H \url{https://doi.org/10.1073/pnas.0711034105}

Blp1 1H \url{https://doi.org/10.3389/fpls.2017.01414}
